\section{Schedule and Milestones}
\label{s:plan}

\if 0
\begin{table}[t!]
\centering
\resizebox{1.0\columnwidth}{!}{
 \input{fig/tbl-plan}
}
\quad\checkmark: important milestone,
\quad\colorbox{gray}{\phantom{XX}} : confirmative plan, 
\quad\colorbox{lightgray}{\phantom{XX}} : tentative plan
\caption{Schedule and milestones to design, implement, and evaluate
  \sys, with the goal of releasing \sys as open source and providing
  it as popular distribution packages.}
\label{t:plan}
\end{table}
\fi

In this section, we lay out important
milestones and corresponding schedules for each year.

\textbf{In year 1}, we will build an initial prototype framework
that integrates basic application C/R and in-place kernel switch.
%
This prototype will serve as the foundation
for exploring research ideas in subsequent years.
%
In parallel, we will explore the performance
optimizations envisioned for application C/R\@.

\textbf{In year 2}, we will extend the prototype framework
to two update validation schemes:
dry-run update and replication-based verification.
%
We will also implement a set of benchmarks to evaluate the
performance, usability, and robustness of \sys's approach
to system updates.
%
Concurrently, we will introduce \sys's approach to
security labs and training materials for the Gatech CTF team.

\textbf{In year 3}, we will explore design choices
as native infrastructure of a commodity OS,
Linux, to efficiently support \sys,
based on the experiences from previous years.
%
In parallel, we will make the prototype of \sys publicly available and
contribute to \criu and \cc{Linux} communities.
%
At this milestone, we will have a clearer idea of which design choices
are more suitable to which environments or under which constraints.

\textbf{In year 4}, we will explore \sys's approach 
in the context of Avionic software 
that has hard real-time constraints and
strict safety requirements.
%
In particular,
we will devise a safe fallback and update validation mechanisms
specialized for avionic software including drone systems,
and analyze their potential limitations
(e.g., types of failures to be handled)
by closely partnering with Boeing.

\textbf{In year 5}, we will extend \sys's approach further to
two other domains:
Energy Delivery System with GTRI
and Cloud infrastructure with Parallels.
%
The primary goal of both domains is to completely
remove the downtime during system updates
by co-designing the full software stack,
unlike typical computing workstation environments.

This project will involve one PI and two PhD students
including one funded by the PI's startup,
and will closely collaborate with
Boeing, GTRI, Parallels and the CRIU open source team.
%
If successful, the developed techniques, experiences,
and open source tools
will not only make a direct impact to computer users,
but also open new opportunities in other domains
where system updates are critical to their security and operation.

%
% Note that our ultimate goal is
% to make a great impact, not only helping end-users and system admins
% update operating systems quickly and safely, but also
% and fostering research communities with inspiring
% technical contributions; our team will spend non-negligible amount of
% time to maintain our project as opensource, matured and maintained
% with \criu communities.
