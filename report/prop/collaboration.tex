\section*{\centerline{Collaboration Plan}}
\label{sec:collaboration}
\vskip-0.15in\rule{6.5in}{0.0005in}

\noindent
The two investigators at Georgia Tech will conduct this proposed research.
They have a wide range of expertise on operating systems,
distributed systems, synchronization, security, programming languages
and computer architecture. Both PIs share strong interests in computer
systems, and more specifically have unique expertise in the following
topics in order to complete all aspects of the proposed work:

\begin{itemize}
\item \textbf{PI Min:}
  Operating system design, manycore operating systems,
  parallel runtime, synchronization, concurrent data structure,
  and storage system.

\item \textbf{PI Kim:}
  Operating system design, manycore operating systems, distributed
  system, security, programming languages, and computer architecture.
\end{itemize}

\vspace{3px}
\PP{PI Roles and Responsibilities}
The project will be directed by PI Min, who will be responsible for
scheduling team meetings, submitting NSF reports, and other managerial
activities.
%
PI Min will oversee all the research and education activities. In
particular, he will lead the development of the idle profiler and the
fabric profiler. He will work closely with PI Kim to integrate efforts
and research results.
%
PI Kim will lead the development of the causal profiler and
FPGA-implemented PMU for the fabric profiler. He will oversee
incorporating results from this project with our education and
curriculum development efforts supported by Intel and NSF.

\vspace{1em}
\PP{Past Collaboration.}
Both PI Min and PI Kim have recently started collaborating on building
scalable operating systems for scale-up architectures. Two notable works
include a scalable synchronization primitive, called
OTicket~\cite{kashyap:oticket, kashyap:oppspinlocks}, and a
scalability benchmark for file systems, called
FxMark~\cite{min:fxmark}. The outcomes of these two projects are
remarkable. The new spinlock design proposed by OTicket was recently
adopted by the mainline Linux kernel~\cite{waiman:spinlock}, and
the FxMark benchmark was recently adopted by the SUSE's QoS team
as part of their testing procedure. These collaborations so far have
helped bridge the different specialties resulting in co-advising
students and the design and implementation of new tools and systems. For
the last two years, they have co-authored seven papers.

\vspace{1em}
\PP{Project Management across Investigators, Institutions and
  Disciplines.}
As a group, the team will develop research plans for the parts of the
work for which they are responsible. Through the coordination
mechanisms described in the next section, both PIs will ensure that
research progresses as planned.

\vspace{1em}
\PP{Specific Coordination Mechanisms.}
Their coordination has been facilitated by being at the same
institution, in the same department, the School of Computer Science
(SCS). Their close physical proximity has been found to be maximally
conducive for successful collaboration~\cite{Olson:2000:DM}. Both PIs
and their students have offices in the same building (3rd floor),
where everyone is within each other's convenient reach. The PIs have
been holding
and will continue to hold weekly project meetings, which will include
both the students and faculty working on this project.
These meetings have been ongoing since last year, and will continue in
a collocated location. The PIs will jointly supervise graduate students.
PI Min and PI Kim are already co-advising three Ph.D. students and run
a joint weekly group meeting. The team already has a \cc{git} repository at
Georgia Tech to host resources, including reference papers, design
documents, source code, and datasets.


\vspace{1em}
\PP{Budget Line Items Supporting Coordination Mechanisms.}
No additional budget items are required to support coordination, as
meetings will be held at Georgia Tech, where all PIs and students are physically
present.
