\section{Evaluation and Management Plan}
\label{sec:manage}
We anticipate this project will require three years to complete. In
this section, we outline important milestones and corresponding
evaluation plan. The details on how the two PIs collaborate and manage
the project are presented in the attached \textbf{Collaboration
Plan}.

\PP{Equipment and Infrastructure.}
The Georgia Tech Information Security Center (GTISC), which the two PIs
belong to, will provide the necessary equipment and infrastructure to
successfully execute this project.
In particular, we will use two large Intel Xeon servers (80-core and
192-core) to develop and test the idle profiler. For
developing and testing the causal profiler, we will use our 16-node
cluster connected with a gigabit Ethernet, and we will use
CloudLab~\cite{cloudlab:web} one of the public clouds to test
larger-scale deployment. For the fabric profiler, we will use two
fabric interconnects supporting 40Gbps and 100Gbps.

\PP{Milestones and Evaluation Plan.}
Each year, we plan to complete one research thrust, as well as
demonstrate its capability by finding scalability bottlenecks in
real-world software. We will release the software from each research
thrust for wide dissemination of technologies to the research community
and relevant industry.

\squishlist
\item \textbf{Year 1. Idle profiling for scale-up architectures.
  {[R1]}} \\
  We will explore several design choices for the idle profiler, including
  discovering spin-loops in binary files, low-overhead dynamic
  profiling, and analyzing profiled results to suggest optimization
  candidates. At the same time, we will develop the prototype for the Linux/x86_64
  platform.
  %
  To evaluate its effectiveness, we will test it
  with various real-world software including two manycore scalability
  benchmark suites developed by the two PIs, FxMark~\cite{min:fxmark} and
  VBench~\cite{kashyap:oppspinlocks}, to stress OS kernel and user
  applications, respectively. We will test it on the two large Intel Xeon
  servers.

\item \textbf{Year 2. Causal profiling for scale-out architectures.
  {[R2]}} \\
  We will explore design choices for the causal profiler including
  emulating a speedup of a sub-task, effective reduction of search
  space, and suggesting optimization candidates. We
  will develop our prototype giving more
  emphasis on integration and testing with popular distributed
  frameworks, such as Apache Spark~\cite{spark} and
  Hadoop~\cite{hadoop:web}, for broader impact.
  % We expect that the research on suggesting optimization candidates
  % can be applicable to the idle profiler as well.
  %
  We will evaluate the causal profiler with realistic cloud workload
  including CloudSuite~\cite{cloudsuite:asplos12} on our cluster,
  CloudLab~\cite{cloudlab:web}, and a public cloud.

\item \textbf{Year 3. Fabric profiling for rack-scale architectures.
  {[R3]}} \\
  We will explore design choices for the fabric profiler including
  monitoring memory hotspot and latency spikes using a NIC-attached
  FPGA and aggregation of profiled data to provide a holistic view of
  bottlenecks in a fabric network. We will develop our prototype for the
  InfiniBand/RDMA network and will initially focus on sharing memory.
  %
  Since there is no publicly available benchmark suite for rack-scale
  architectures, we will also develop a rack-scale benchmark suite comprising
  of popular RDMA applications (e.g., DrTM~\cite{drtm:sosp15},
  Grappa~\cite{grappa:atc15}, RAMCloud~\cite{john:ramcloud}).
  %We will test it on our rack-scale system.
\squishend

This project will involve two PIs and one Ph.D. student who will
closely collaborate with corresponding open source communities. If
successful, the developed techniques, experiences, and open source
tools will not only make a direct impact to software developers, but
also accelerate the designing of scalable systems for both the research
community and industry.
