\section*{\centerline{Data Management Plan}}
\vskip-0.15in\rule{6.5in}{0.0005in}

\noindent
Following the NSF Data Management Guidelines,
we will administer, archive, and maintain
the data and the intellectual properties generated during this project.
%
The data acquired, processed, and preserved in the
context of this proposal will be further governed by policies of
Georgia Institute of Technology pertaining to intellectual property,
record retention, and data management.
%
In this document,
we describe the expected data that we will generate and collect,
as well as a detailed plan to openly manage them
without violating the privacy of users.

\vspace{10px}
\subsection*{1.~~Expected Data}

\squishlist
\item Source code of our scalability profilers:
  idle profiler, causal profiler, and fabric profiler;
\item Details of experiment settings and analysis results in support of reproducible science;
\item Package distribution and releases of our scalability profilers;
\item Publications: academic papers, technical manuals and blog postings;
\item Curriculum materials: tutorials and lab materials for operating
  system and distributed systems courses.
\squishend

\noindent
As our goal is to generate a tangible influence on our society;
all the developed techniques and implementation,
even generated via collaboration with other domain experts,
will be kept publicly accessible,
and ultimately tech-transferred to the Linux Foundation and the Apache
Foundation for future maintenance.

\vspace{10px}
\subsection*{2.~~Data Storage and Archiving Policies}

The Georgia Tech Information Security Center (GTISC), where the PIs are
mainly involved in, has an operation team
with multiple permanent system administrators
who are specialized in maintaining our group's infrastructure.
%
The central repository (e.g., gitolite, code review system,
monitoring, and backup services) is protected by RAID5
in local storage, as well as backed up daily to
a dedicated archiving server.
%
Our internal services and machines are regularly audited
by security experts
and updated to meet the standard security policy of GTISC.

\vspace{10px}
\subsection*{3.~~Open Source Initiative and Data Dissemination}

Our developed techniques, and implementations, as part of this
proposal, will be continuously released and kept publicly available
to reach out to broader communities.

\PP{Source code.}
The prototype that is built on top of open source project (e.g., GPL)
will be correspondingly licensed using the same policy. If the
software is entirely developed from scratch without using any other
proprietary software, the PIs will make it open source with detailed
documentation using one of the open source license policies. In
particular, we plan to make an early release of our research prototype
via our group's GitHub repository.

\begin{verbatim}
    $ git clone https://github.com/sslab-gatech/
\end{verbatim}

\PP{Experiment results.}
We will maintain a set of experiment scripts that will let anyone
reuse our intermediate results or completely reproduce them from
scratch.
%
All data, technical reports, peer-reviewed publication, and package
distributions will be provided on the project and group web sites.
