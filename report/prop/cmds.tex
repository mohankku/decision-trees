\renewcommand{\ttdefault}{pxtt}

\usepackage{amsthm}
\newcommand{\URL}{\url}
\newcommand{\cc}[1]{\mbox{\smaller[0.5]\texttt{#1}}}

%\clubpenalty=10000
%\widowpenalty=10000

%\linespread{1.2}

\fvset{fontsize=\scriptsize,xleftmargin=8pt,numbers=left,numbersep=5pt}

\input{code/fmt}
\newcommand{\figrule}{\hrule width \hsize height .33pt}
\newcommand{\coderule}{\vspace{-0.4em}\figrule}

\setlength{\abovedisplayskip}{0pt}
\setlength{\abovedisplayshortskip}{0pt}
\setlength{\belowdisplayskip}{0pt}
\setlength{\belowdisplayshortskip}{0pt}
\setlength{\jot}{0pt}

\def\Snospace~{\S{}}
\renewcommand*\sectionautorefname{\Snospace}
\def\sectionautorefname{\Snospace}
\def\subsectionautorefname{\Snospace}
\def\subsubsectionautorefname{\Snospace}
\def\chapterautorefname{\Snospace}
%\renewcommand{\figurename}{Fig.}
%\def\figureautorefname{\figurename}
\newcommand{\subfigureautorefname}{\figureautorefname}

%\numberwithin{equation}{section}
\newcommand{\yes}{Y}
\newcommand{\no}{}

% sema
\newcommand{\shl}{\ \cc{<}\cc{<}\ }
\newcommand{\shr}{\ \cc{>}\cc{>}\ }

\if 0
\renewcommand{\topfraction}{0.9}
\renewcommand{\dbltopfraction}{0.9}
\renewcommand{\bottomfraction}{0.8}
\renewcommand{\textfraction}{0.05}
\renewcommand{\floatpagefraction}{0.9}
\renewcommand{\dblfloatpagefraction}{0.9}
\setcounter{topnumber}{10}
\setcounter{bottomnumber}{10}
\setcounter{totalnumber}{10}
\setcounter{dbltopnumber}{10}
\fi

\newif\ifdraft\drafttrue
\newif\ifnotes\notestrue
\ifdraft\else\notesfalse\fi

% ref. http://en.wikibooks.org/wiki/LaTeX/Colors
\newcommand{\TK}[1]{\textcolor{LimeGreen}{TK: #1}}
\newcommand{\SK}[1]{\textcolor{blue}{SK: #1}}
\newcommand{\MK}[1]{\textcolor{red}{MK: #1}}
\newcommand{\SM}[1]{\textcolor{CornflowerBlue}{SM: #1}}
\newcommand{\WK}[1]{\textcolor{BurntOrange}{WK: #1}}
\newcommand{\CM}[1]{\textcolor{Violet}{CM: #1}}
\newcommand{\XXX}[1]{\textcolor{red}{XXX: #1}}
\newcommand{\X}[0]{\textcolor{red}{XXX}}
\newcommand{\TODO}[1]{\textcolor{Melon}{TODO: #1}}

\newtheorem{Challenge}{\textbf{Research Challenge}}
% hide comments
% \renewcommand{\TK}[1]{\ignorespaces}
% \renewcommand{\XXX}[1]{\ignorespaces}
% \renewcommand{\TODO}[1]{\ignorespaces}

%% Ensure ligatures (e.g., ``fine official flag'') can be copy/pasted from PDF.
\input{glyphtounicode}
\pdfgentounicode=1

\newcolumntype{R}[1]{>{\raggedleft\let\newline\\\arraybackslash\hspace{0pt}}p{#1}}

% include macros
\newcommand{\includepdf}[1]{
  \includegraphics[width=\columnwidth]{#1}
}
\newcommand{\includeplot}[1]{
  \resizebox{\columnwidth}{!}{\input{#1}}
}

% list
\newcommand{\squishlist}{
\begin{itemize}[noitemsep,nolistsep]
  \setlength{\itemsep}{-0pt}
}
\newcommand{\squishend}{
  \end{itemize}
}

%%
%% NOTE.
%%  to use circled number in caption, use
%%   (e.g., \protect\C{1})
%%
\usepackage{tikz}
\newcommand*\C[1]{%
\begin{tikzpicture}[baseline=(C.base)]
\node[draw,circle,inner sep=0.2pt](C) {#1};
\end{tikzpicture}}

\newcommand*\BC[1]{%
\begin{tikzpicture}[baseline=(C.base)]
\node[draw,circle,fill=black,inner sep=0.2pt](C) {\textcolor{white}{#1}};
\end{tikzpicture}}

\newcommand{\PP}[1]{
%\vspace{2px}
\noindent{\bf #1}
}

\setlength{\parskip}{2px}
\setlist{itemsep=1pt,parsep=1pt,topsep=1pt,partopsep=1pt}
\newcommand{\V}{\checkmark}
\newcommand{\x}{$\times$\xspace}

%% new commands
% tools
\newcommand{\perf}{\mbox{\cc{perf}}\xspace}
\newcommand{\kprobe}{\mbox{\cc{kprobe}}\xspace}

% media
\newcommand{\mem}{\mbox{\cc{RAMDISK}}\xspace}
\newcommand{\ssd}{\mbox{\cc{SSD}}\xspace}
\newcommand{\hdd}{\mbox{\cc{HDD}}\xspace}

% workloads
\newcommand{\exim}{\mbox{Exim}\xspace}
\newcommand{\rocksdb}{\mbox{RocksDB}\xspace}
\newcommand{\leveldb}{\mbox{LevelDB}\xspace}
\newcommand{\mongodb}{\mbox{MongoDB}\xspace}
\newcommand{\filebench}{\mbox{Filebench}\xspace}
\newcommand{\dbench}{\mbox{\smaller[.8]DBENCH}\xspace}
\newcommand{\dbbench}{\texttt{db\_bench}\xspace}

% file system
\newcommand{\btrfs}{\mbox{\cc{btrfs}}\xspace}
\newcommand{\ext}{\mbox{\cc{ext4}}\xspace}
\newcommand{\extn}{\mbox{\cc{ext4$_{\text{NJ}}$}}\xspace}
\newcommand{\ffs}{\mbox{\cc{F2FS}}\xspace}
\newcommand{\tmpfs}{\mbox{\cc{tmpfs}}\xspace}
\newcommand{\xfs}{\mbox{\cc{XFS}}\xspace}

\newcommand{\jbd}{\mbox{\cc{JBD2}}\xspace}
\newcommand{\vfs}{\mbox{\cc{VFS}}\xspace}

\newcommand{\boxbeg}{
\noindent
\begin{tabular}{|l|}\hline
\begin{minipage}{6.3in}
\vspace{0px}
\noindent \it
}

\newcommand{\boxend}{
\vspace{0px}
\end{minipage}\\ \hline
\end{tabular}
\vspace{3px}
}
